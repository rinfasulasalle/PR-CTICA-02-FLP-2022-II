\section{Ejercicios}

Lea el libro “Programming Language Pragmatics” Scott (2000), Capitulo 1, Secci ́on 1.1 - 1.6
y responda las siguientes preguntas:

\subsection{¿Cuál es la diferencia entre lenguaje de máquina y lenguaje ensamblador?}

El lenguaje de máquina se usaba antes, en  los primeros ordenadores, los que eran del porte de una habitación y costaba mucho dinero, el lenguaje de máquina es la secuencia de bits que controla directamente un procesador. Con el paso del tiempo, se crearon los lenguajes ensambladores que eran menos propensos a errores y estos permiten que las operaciones  se expresaran con abreviaturas nemotécnicas.


\subsection{¿En qué circunstancia un lenguaje de alto nivel es superior al lenguaje ensamblador?}

Principalmente, los lenguajes de alto nivel están orientados al software, son fáciles de entender, requieren de menos instrucciones para cada tarea y estos nos permiten crear aplicaciones.

\subsection{¿En qué circunstancia un lenguaje ensamblador es superior al lenguaje de alto nivel?}

Cuando se trata de la construcción de sistemas operativos  y del kernel, además este lenguaje facilita la comprensión de cómo es que funciona cada instrucción en la computadora.


\subsection{¿Por qué hay tantos lenguajes de programación? }

Una de las principales causas de la gran variedad de lenguajes de programación es de que muchos se han diseñado para problemas específicos, por ejemplo prolog que sirve para conocer las relaciones lógicas entre datos. 

Además de ello, también se toma en cuenta la preferencia personal, a algunos les gusta la brevedad C y otras la odian, algunos prefieren las iteraciones sobre la recursividad, etc.

También destaca si es fácil de aprender su sintaxis o no, por ejemplo algunos prefieren Python sobre java porque es más sencillo y más fácil de entender si eres nuevo en el tema.

La facilidad de implementación también es un factor a considerar ya que si tenemos recursos limitados, al momento de elegir el lenguaje de programación nos encontraremos con algunas limitaciones que podrían cambiar nuestras preferencias de lenguaje a utilizar.

Por último, la publicidad que tiene cada lenguaje es importante ya que si algo es conocido o se ha escuchado del mismo por ley da confianza a probarlo, sin embargo si queremos aventurarnos en un lenguaje que no hayamos escuchado nunca, estamos algo perdidos ya que no sabemos nada ni hemos oído nada.



\subsection{Nombre 3 lenguajes en las categorías de von Neumann, funcional y orientado a objetos. Dos lenguajes lógicos y 2 concurrentes.}

3 lenguajes en la categoría de von Neumann: Smalltalk, eiffel, C++, Java.

 3 lenguajes en las categorías de funcional: LISP, Haskell, Scala.
 
3 lenguajes en las categorías de orientado a objetos:Java, JavaScript, C++.

Dos lenguajes lógicos: ProLog, Erlang.

Dos lenguajes concurrentes: Ada, Posix.


\subsection{¿Que distingue a los lenguajes declarativos e imperativos?}

Imperativo, como una secuencia de operaciones a realizar. Declarativo, se especifica el resultado deseado, no cómo lograrlo.


\subsection{¿Cuál es considerado el primer lenguaje de alto nivel? }

En la década de 1940 fueron creadas las primeras computadoras modernas, con alimentación eléctrica. La velocidad y capacidad de memoria limitadas forzaron a los programadores a escribir programas, en lenguaje ensamblador muy afinados. Finalmente se dieron cuenta de que la programación en lenguaje ensamblador requería de un gran esfuerzo intelectual y era muy propensa a errores. Es entonces que nace el lenguaje FORTRAN (FORmula TRANslator), que fue desarrollado por IBM en 1954 para mejorar esto.


\subsection{¿Cuál es considerado el primer lenguaje funcional? }

LISP (1958), creado por John McCarthy.

\subsection{¿Por qué los lenguajes concurrentes no están considerados en la clasificación de Scott (2000) (Figura 1?1)?}

Por que en ese entonces no se podría ejecutar multitareas con los diferentes lenguajes o métodos
de programación, aparte de que utilizan diferentes tipos de paquetes o bibliotecas


\subsection{Enlista las principales fases de un compilador y describe la función de cada fase}

respuesta

\subsection{¿En qué circunstancias tiene sentido que un compilador pase o revise el código varias veces?}

Cuando se cree que se cometio un error.

\subsection{¿Cuál es el propósito de la tabla de símbolos en un compilador?}

Sirve para guardar la información de lo que le ocurre a diversas entidades, variables, nombres, objetos,clases, interfazes graficas, etc..

\subsection{¿En la actualidad, que programa es más eficiente, uno desarrollado desde cero en ensamblador o uno generado por un compilador?}

respuesta
